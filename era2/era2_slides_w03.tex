%%
% This is an Overleaf template for presentations
% using the TUM Corporate Desing https://www.tum.de/cd
%
% For further details on how to use the template, take a look at our
% GitLab repository and browse through our test documents
% https://gitlab.lrz.de/latex4ei/tum-templates.
%
% The tumbeamer class is based on the beamer class.
% If you need further customization please consult the beamer class guide
% https://ctan.org/pkg/beamer.
% Additional class options are passed down to the base class.
%
% If you encounter any bugs or undesired behaviour, please raise an issue
% in our GitLab repository
% https://gitlab.lrz.de/latex4ei/tum-templates/issues
% and provide a description and minimal working example of your problem.
%%

\PassOptionsToClass{onlytextwidth}{beamer}

\documentclass[
  german,            % define the document language (english, german)
  aspectratio=169,    % define the aspect ratio (169, 43)
  % handout=2on1,       % create handout with multiple slides (2on1, 4on1)
  % partpage=false,     % insert page at beginning of parts (true, false)
  % sectionpage=true,   % insert page at beginning of sections (true, false)
]{tumbeamer}


% load additional packages
\usepackage{booktabs}
\usepackage{graphicx}
\usepackage{tikz}
\usepackage{url}
\usepackage{pgfplots}
\usepackage{hyperref}
\usepackage{pmboxdraw}
\usepackage{float}
\usepackage{babel}[ngerman]
\usepackage{csquotes}[autostyle]
\usepackage[useregional]{datetime2}
\usepackage[cache=true]{minted}

% tikz

% minted
\setminted{
    fontsize=\small, 
    frame=single,
    breaklines=true,
}

% image path
\graphicspath{ {./resources/} }

% presentation metadata
\title{Übung 03: Sprünge und Pointer}
\subtitle{Einführung in die Rechnerarchitektur}
\author{Niklas Ladurner}

\institute{\theChairName\\\theDepartmentName\\\theUniversityName}
\date{\DTMdisplaydate{2024}{11}{1}{-1}}

\footline{\insertauthor~|~\insertshorttitle~|~\insertshortdate}


% macro to configure the style of the presentation
\TUMbeamersetup{
  title page = TUM tower,         % style of the title page
  part page = TUM toc,            % style of part pages
  section page = TUM toc,         % style of section pages
  content page = TUM more space,  % style of normal content pages
  tower scale = 1.0,              % scaling factor of TUM tower (if used)
  headline = TUM threeliner,      % which variation of headline to use
  footline = TUM default,         % which variation of footline to use
  % configure on which pages headlines and footlines should be printed
  headline on = {title page},
  footline on = {every page, title page=false},
}

% available frame styles for title page, part page, and section page:
% TUM default, TUM tower, TUM centered,
% TUM blue default, TUM blue tower, TUM blue centered,
% TUM shaded default, TUM shaded tower, TUM shaded centered,
% TUM flags
%
% additional frame styles for part page and section page:
% TUM toc
%
% available frame styles for content pages:
% TUM default, TUM more space
%
% available headline options:
% TUM empty, TUM oneliner, TUM twoliner, TUM threeliner, TUM logothreeliner
%
% available footline options:
% TUM empty, TUM default, TUM infoline


\begin{document}

\maketitle

\begin{frame}[c]{}{}
  \begin{center}
    \LARGE  Keine Garantie für die Richtigkeit der Tutorfolien.

    \Large Bei Unklarheiten/Unstimmigkeiten haben VL/ZÜ-Folien recht!
  \end{center}
\end{frame}

\begin{frame}[c, fragile]{Sprungbefehle}{}
  \vspace{-1cm}
  \begin{columns}
    \begin{column}{0.5\textwidth}
      \begin{center}\textbf{Branch-Befehle}\end{center}
      Rücksprungadresse wird nicht gesichert, \textbf{Sprungbedingung}
      muss erfüllt sein
      \begin{itemize}
        \item \verb|beq|: $rs1 = rs2$
        \item \verb|bne|: $rs1 \ne rs2$
        \item \verb|blt(u)|: $rs1 < rs2$
        \item \verb|bgt(u)|: $rs1 > rs2$
      \end{itemize}
    \end{column}
    \begin{column}{0.5\textwidth}
      \begin{center}\textbf{Jump-Befehle}\end{center}
      Schreiben Rücksprungadresse in ra oder angegebenes Register,
      springen immer
      \begin{itemize}
        \item \verb|jal label|
        \item \verb|jalr rd, offset(rs)|
        \item \verb|j label| Achtung, überschreibt ra nicht!
      \end{itemize}
    \end{column}
  \end{columns}
\end{frame}

\begin{frame}[c, fragile]{Speicherzugriffe}{}
  lade 32 Bit an der Adresse a0 + 0 Bytes Offset in das Register t0:
  \begin{verbatim}lw t0, 0(a0)\end{verbatim}
  lade 8 Bit an der Adresse a2 - 4 Bytes Offset in das Register t1:
  \begin{verbatim}lb t1, -4(a2)\end{verbatim}
  speichere den gesamten Inhalt des Registers t2 an die Adresse a1 + 16 Bytes Offset:
  \begin{verbatim}sw t2, 16(a1)\end{verbatim}
\end{frame}

\begin{frame}[c, fragile]{Sections und Direktiven}
  \begin{columns}[c]
    \begin{column}{0.45\textwidth}
      \begin{minted}{gas}
# compile-time Konstante
.equ NUM, 2748 

# ro + init. Daten
.rodata
  f: .word 2

# rw + init. Daten
.org 0x400
.data
  arr: .byte 4, 3, 2, 1
  string1: .ascii "asdf"
  string2: .asciz "asdf"
        \end{minted}
    \end{column}
    \begin{column}{0.45\textwidth}
      \begin{minted}{gas}
# rw + uninit. Daten
.bss
  a: .space 16

# globales Einstiegslabel
.globl _start

.org 0x200 # section beginnt an Adresse 0x200
.text
_start:
   la a0, arr
   lbu a1, 0(a0)
      \end{minted}
    \end{column}
  \end{columns}
  \begin{center}
  \small{\textbf{ro}: read-only, \textbf{rw}: les- und schreibbar}
  \end{center}
\end{frame}


\begin{frame}[c]{}{}
  \begin{center}
    \LARGE Fragen?\\
    \Large (Die ZÜ-Folien sind sehr gut, schaut euch die an)
  \end{center}
\end{frame}

\begin{frame}[c, fragile]{Artemis-Hausaufgaben}{}
  \begin{itemize}
    \item \enquote{H03 --- Palindromerkennung} bis 10.11.2024 23:59 Uhr
    \item Verwendung von Speicheroperationen, Unterprogrammaufrufe
    \item Pseudoinstruktion \verb|tail| für tailcalls, verwendet aktuellen Stackframe wieder
  \end{itemize}
\end{frame}

\begin{frame}[c, fragile]{Links}{}
  \begin{itemize}
    \item Zulip: \href{https://zulip.in.tum.de/#narrow/stream/2661-ERA-Tutorium---Do-1600-1}{\enquote{ERA Tutorium - Do-1600-1}}
          bzw. \href{https://zulip.in.tum.de/#narrow/stream/2675-ERA-Tutorium---Fr-1500-2 }{\enquote{ERA Tutorium - Fr-1500-2}}
    \item \href{https://riscv.org/wp-content/uploads/2017/05/riscv-spec-v2.2.pdf}{RISC-V-Spezifikation}
    \item \href{https://www.moodle.tum.de/course/view.php?id=100633}{ERA-Moodle-Kurs}
    \item \href{https://artemis.in.tum.de/courses/401}{ERA-Artemis-Kurs}
    \item \href{https://msyksphinz-self.github.io/riscv-isadoc/html/index.html}{Übersicht an RISC-V-Instruktionen}
    \item \href{https://ftp.gnu.org/old-gnu/Manuals/gas/html_chapter/as_7.html}{GNU as directives}
  \end{itemize}
\end{frame}

\maketitle

\end{document}
