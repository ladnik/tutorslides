%%
% This is an Overleaf template for presentations
% using the TUM Corporate Desing https://www.tum.de/cd
%
% For further details on how to use the template, take a look at our
% GitLab repository and browse through our test documents
% https://gitlab.lrz.de/latex4ei/tum-templates.
%
% The tumbeamer class is based on the beamer class.
% If you need further customization please consult the beamer class guide
% https://ctan.org/pkg/beamer.
% Additional class options are passed down to the base class.
%
% If you encounter any bugs or undesired behaviour, please raise an issue
% in our GitLab repository
% https://gitlab.lrz.de/latex4ei/tum-templates/issues
% and provide a description and minimal working example of your problem.
%%


\documentclass[
  german,            % define the document language (english, german)
  aspectratio=169,    % define the aspect ratio (169, 43)
  % handout=2on1,       % create handout with multiple slides (2on1, 4on1)
  % partpage=false,     % insert page at beginning of parts (true, false)
  % sectionpage=true,   % insert page at beginning of sections (true, false)
]{tumbeamer}


% load additional packages
\usepackage{booktabs}
\usepackage{graphicx}
\usepackage{tikz}
\usepackage{url}
\usepackage{pgfplots}
\usepackage{hyperref}
\usepackage{pmboxdraw}
\usepackage{float}

% image path
\graphicspath{ {./resources/} }

% presentation metadata
\title{Übung 06: SIMD in x86 und C}
\subtitle{Grundlagenpraktikum Rechnerarchitektur}
\author{Niklas Ladurner}

\institute{\theChairName\\\theDepartmentName\\\theUniversityName}
\date[\today]{\today}

\footline{\insertauthor~|~\insertshorttitle~|~\insertshortdate}


% macro to configure the style of the presentation
\TUMbeamersetup{
  title page = TUM tower,         % style of the title page
  part page = TUM toc,            % style of part pages
  section page = TUM toc,         % style of section pages
  content page = TUM more space,  % style of normal content pages
  tower scale = 1.0,              % scaling factor of TUM tower (if used)
  headline = TUM threeliner,      % which variation of headline to use
  footline = TUM default,         % which variation of footline to use
  % configure on which pages headlines and footlines should be printed
  headline on = {title page},
  footline on = {every page, title page=false},
}

% available frame styles for title page, part page, and section page:
% TUM default, TUM tower, TUM centered,
% TUM blue default, TUM blue tower, TUM blue centered,
% TUM shaded default, TUM shaded tower, TUM shaded centered,
% TUM flags
%
% additional frame styles for part page and section page:
% TUM toc
%
% available frame styles for content pages:
% TUM default, TUM more space
%
% available headline options:
% TUM empty, TUM oneliner, TUM twoliner, TUM threeliner, TUM logothreeliner
%
% available footline options:
% TUM empty, TUM default, TUM infoline


\begin{document}

\maketitle

\begin{frame}[c]{}{}
  \begin{center}
    \LARGE  Keine Garantie für die Richtigkeit der Tutorfolien: Bei Unklarheiten/Unstimmigkeiten
    haben VL-Folien Recht!
  \end{center}
\end{frame}

\begin{frame}[c, fragile]{SIMD}{}
  \begin{itemize}
    \item \textbf{S}ingle \textbf{I}nstruction, \textbf{M}ultiple \textbf{D}ata
    \item Vektorisierung von Instruktionen, parallele Ausführung auf mehreren Datenströmen
    \item Achtung: SIMD ist nicht immer sinnvoll / einfach zu implementieren!
    \item XMM-Register mit 128 Bit (16 Byte), bspw.
    \begin{itemize}
      \item 16 chars (1 Byte)
      \item 4 ints (4 Byte)
      \item 4 floats (4 Byte)
      \item 2 doubles (8 Byte)
    \end{itemize}
    \item Naming Convention: p für '\textbf{p}acked' (vgl. 'scalar')
    \item Achtung: Bei SIMD-Instruktionen wird 16-Byte-Alignment gefordert! \\ $\rightarrow$ \verb|movaps| vs. \verb|movups|
  \end{itemize}
\end{frame}

\begin{frame}[c, fragile]{SIMD in C}{}
  \begin{itemize}
    \item Nutzung von sog. \textit{Intrinsics} (Liste siehe Intel Intrinsics Guide)
    \item im Praktikum werden SSE/SSE2/AVX verwendet $\rightarrow$ muss von der Zielarchitektur unterstützt werden
    \item neue Datentypen \verb|__m128i|, \verb|__m128d|, \ldots
    \item Funktionen für einzelne SIMD-Operationen
    \item \verb|#include <immintrin.h>| nicht vergessen!
    \item Vergleiche in SIMD ergeben Bitmasken
  \end{itemize}
\end{frame}

\begin{frame}[c]{}{}
  \begin{center}
    \LARGE Fragen?
  \end{center}
\end{frame}

\begin{frame}[fragile, c]{Links}{}
  \begin{itemize}
    \item Zulip: \href{https://zulip.in.tum.de/#narrow/stream/2267-GRA-Tutorium---Gruppe-20}{\glqq GRA Tutorium - Gruppe 20\grqq}
          bzw. \href{https://zulip.in.tum.de/#narrow/stream/2269-GRA-Tutorium---Gruppe-22}{\glqq GRA Tutorium - Gruppe 22\grqq}
          bzw. \href{https://gra.caps.in.tum.de}{\glqq Praktikumswebsite\grqq}
    \item \href{https://www.felixcloutier.com/x86/}{x86 instruction reference by Félix Cloutier}
    \item \href{https://www.intel.com/content/www/us/en/docs/intrinsics-guide/index.html#}{Intel Intrinsics Guide}
  \end{itemize}
\end{frame}

\maketitle

\end{document}
